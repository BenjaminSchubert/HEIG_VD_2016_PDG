\documentclass[french]{article}
\usepackage[T1]{fontenc}
\usepackage[utf8]{inputenc}
\usepackage{lipsum}
\usepackage{lmodern}
\usepackage{geometry}
\usepackage{babel}
\usepackage{graphicx}
\usepackage{lastpage}
\usepackage{ragged2e}

\geometry{
	a4paper,
	total={210mm,297mm},
	left=20mm,
	right=20mm,
	top=20mm,
	bottom=20mm,
}

\usepackage{fancyhdr}
\pagestyle{fancy}

\lhead{Champion, Loiseau\\Rochat, Schubert}
\chead{Cahier des charges}
\rhead{\today}
\renewcommand{\headrulewidth}{0.4pt}
\renewcommand{\footrulewidth}{0.4pt}

\begin{document}
	\centering
	\large{\textbf{PDG: Cahier des charges}}
	
	\justify
	
	\section{Motivation}
		 Lors d'un fesival il est courrant de perdre ses amis en revanche les %
		 retrouver est souvent moins aisé. 
	
	\section{But et réalisation}
		L'application est basée sur trois grands pôles de développement: la gestion de la liste d'amis, la géolocalisation et la sécurité des données utilisateurs.		
	
	\section{Fonctionalités}
	
		bloquer les alertes / demande pendant un certain temps \\
		pas de groupe \\
		spécification du workflow specifier les specification en cas de blocquage etc... \\
	
		\subsection{Gestion de la liste d'amis}
		
			\begin{itemize}
				\item \textbf{Ajout de contact : } L'utilisateur peut ajouter un contact celuici sera inscrit dans la base de donnée et l'autre utilisateur doit accepter l'autre utilisateur avant qu'il soit ajouté dans la liste
				\item \textbf{suppression de contact : } L'utilisateur supprime un contact ...
				\item \textbf{bloquer contact : }
				\item \textbf{recherche d'amis dans la liste de contact du mobile : }
				\item \textbf{Ajout a partir d'un QR Code : }
				
				\item \textbf{création d'un battletag : } bob  1234
				
				\item \textbf{Historique : } L'utilisateur pourra voir ses anciens groupes et ancienne demande 
				
				
				
				
				
			\end{itemize}
		
		\subsection{Securité}
			\begin{itemize}
				\item \textbf{Hash des données sensible : } numero de tel
				\item \textbf{Chiffrage de la communication Client-Serveur / Serveur-DB: } SSL
				\item \textbf{ Gestion restrictive des droits d'accès : } Qui peut demander quoi au serveur ...
				\item \textbf{Diffusion des donnée : } Option de durée d'autorisation par personne (Toujours, 1h, 4h, 1j, 1an, 1min, 1ns)
			\end{itemize}
		\subsection{Géolocalisation}
			\begin{itemize}
				\item \textbf{Point fixe fixée par l'utilisateur : }
				\item \textbf{Sur une personne : }
				\item \textbf{Meilleur point fixe calculé : }
			\end{itemize}
			
		
		\subsection{Interface}
			\begin{itemize}
				\item \textbf{Interface d'administration : } Accès pour l'administrateur aux données ...
				\item \textbf{Interface utilisateur : }
				\begin{itemize}
					\item boussole
					\item radar
					\item visualisation de la map
					\item design 
				\end{itemize} 
			\end{itemize}
		
		\subsection{Base de donnée et accès aux données}
		
		Ce projet est très intéressant sous plusieurs aspects. Premièrement nous aurions l'occasion de découvrir de nouvelles technologies dans le cadre du développement mobile, deuxièmement il s'agit d'une application pouvant réellement être utile et pour finir les possibilités d'extension sont nombreuses (boussole en mode réalité augmentée, chat intégré, etc.).
		
	\section{Technologies}
		Nous utiliseront :
		\begin{itemize}
			\item Google Cloud Messaging
			\item Meteor
			\item Open Street Map
			\item gyroscope
			\item géolocalisation
			\item Technologie android
			\item Base de donnée SQL
			\item Django
			\item HTML/CSS
			\item SSL ?
			\item ...
		\end{itemize}
		
	\section{Annexe}
		
	
\end{document}
