\documentclass[french]{article}
\usepackage[T1]{fontenc}
\usepackage[utf8]{inputenc}
\usepackage{lipsum}
\usepackage{lmodern}
\usepackage{geometry}
\usepackage{babel}
\usepackage{graphicx}
\usepackage{lastpage}
\usepackage{ragged2e}

\geometry{
	a4paper,
	total={210mm,297mm},
	left=20mm,
	right=20mm,
	top=20mm,
	bottom=20mm,
}

\usepackage{fancyhdr}
\pagestyle{fancy}

\lhead{Champion, Loiseau\\Rochat, Schubert}
\chead{Cahier des charges}
\rhead{\today}
\renewcommand{\headrulewidth}{0.4pt}
\renewcommand{\footrulewidth}{0.4pt}

\begin{document}
	
	\title{Projet Rady - Cahier des charges}
	\author{Projet de groupe\\
		Champion, Loiseau, Rochat, Schubert\\
		Resp. René Rentsch\\
		HEIG-VD}
	\date{\today}
	\maketitle
	
	\tableofcontents
	
	\justify
	
	\section{Motivation}
		 Nous souhaitons réaliser une application mobile permettant de se retrouver facilement entre amis. L’idée nous est venue des festivals de cet été, dans lesquels il est parfois compliqué de se retrouver au milieu d’une grande foule. L’application se présenterait sous la forme d’une liste d’amis, laquelle permettrait de créer des groupes et de lancer une géolocalisation. L’utilisateur aurait alors le choix entre une carte en vue du dessus ou une boussole indiquant la direction à
		 prendre.
	
	\section{But et réalisation}
		% Comment notre application va répondre à la problématique?
		% expliquer le concept de groupe, de liste d'amis, de retrouvailles, etc. afin de bien comprendre la suite du document
		% utiliser le mot "retrouvaille" ou un autre ?
		% Plateformes cibles
	
	\section{Fonctionnalités}
		% ajouter des (optionnelle)s ?
	
		\subsection{Compte utilisateur}
			\begin{itemize}
				\item Au premier lancement de l'application, chaque utilisateur sera invité à fournir un pseudonyme, un email ainsi qu'un mot de passe afin de \textbf{créer un compte} 
				\item Chaque pseudonyme sera accompagné d'un numéro afin de permettre à plusieurs utilisateurs d'avoir le \textbf{même pseudonyme}
				\item Un utilisateur pourra \textbf{se déconnecter}, \textbf{changer son numéro de téléphone} ou encore \textbf{supprimer son compte}
				\item Si un utilisateur oublie son mot de passe, une \textbf{ré-initialisation} sera possible grâce à son email
			\end{itemize}
	
		\subsection{Liste d'amis}
			\begin{itemize}
				\item L'utilisateur pourra \textbf{ajouter un ami} à sa liste, lequel devra accepter la demande de son côté
				\item L'utilisateur pourra \textbf{supprimer un ami} ou \textbf{bloquer un ami} de sa liste
				\item Il sera possible de \textbf{rechercher des amis} depuis la liste de contacts du téléphone
				\item Chaque utilisateur aura un \textbf{QR Code unique}, lequel pourra être scanné (via l'appareil photo) par un autre utilisateur afin d'effectuer un ajout rapide à la liste d'amis 
			\end{itemize}
		
		\subsection{Sécurité}
			\begin{itemize}
				\item \textbf{Les champs de la base de données} ne devant pas être lus directement \textbf{seront hashés avec un sel} (par ex: mot de passe, numéro de téléphone)
				\item \textbf{Les communications} entre les clients et le serveur, ainsi qu'entre le serveur et la base de données, \textbf{seront chiffrées} à l'aide de SSL
				\item Les utilisateurs ne pourront pas demander au serveur des informations auxquels ils n'ont pas les \textbf{droits} 
				\item Un utilisateur peut se mettre en \textbf{invisible} pour une certaine durée durant laquelle ses amis ne pourront pas lui envoyer des demandes de retrouvailles
				\item Un utilisateur pourra toujours \textbf{accepter ou refuser} une demande de retrouvailles, celle-ci lui indiquera également si un \textbf{ami bloqué} se trouve dans le groupe
			\end{itemize}
		
		\subsection{Retrouvailles}
			\begin{itemize}
				\item La création d'un groupe de retrouvailles sera simple et rapide pour l'utilisateur
				\item Une retrouvaille pourra se faire sur un \textbf{lieu fixé} à l'avance 
				\item Une retrouvaille pourra se faire \textbf{sur une personne} du groupe, cela implique que la position peut se déplacer
				\item Une retrouvaille pourra chercher le \textbf{meilleur lieu de rencontre} (fixé) afin de minimiser la distance à parcourir par chacune des personnes d'un groupe
				\item L'application sera munie d'une partie \textbf{historique} pour lancer d'anciennes retrouvailles rapidement 
				\item L'interface lors des retrouvailles pourra s'afficher sous différentes formes: \textbf{boussole, radar} ou encore \textbf{carte en vue du dessus}
			\end{itemize}
			
		\subsection{Interface}
			\begin{itemize}
				\item Une \textbf{interface administrateur} offrira la possibilité de modifier la base de données ainsi que des \textbf{statistiques} sur celle-ci
				\item L'\textbf{interface utilisateur}:
				\begin{itemize}
					\item sera \textbf{claire, simple et rapide}, avec peu de menus et d'actions possibles
					\item aura une \textbf{gestion du jour et de la nuit} afin d'offrir les meilleurs contrastes possibles (l'utilisateur aura le choix via les options: jour, nuit, ou automatique)
				\end{itemize}
			\end{itemize}
		
	\section{Annexes}
		\subsection{Technologies}
			% mettre en forme, et dire pourquoi cela va être un défi pour nous
			Nous pensons utiliser les technologies suivantes:
			\begin{itemize}
				\item \textbf{Firebase Cloud Messaging} -
				\item \textbf{MeteorJS} -
				\item \textbf{Open Street Map} -
				\item \textbf{Mobile} (gyroscope, géolocalisation) - 
				\item \textbf{Base de donnée SQL} -
				\item \textbf{Python/Django} - \\
			\end{itemize}
		
			Ces technologies nous sont majoritairement inconnues et seront donc un défi, tant au niveau de l'apprentissage que de la réalisation. La plupart feront l'objet d'un test de faisabilité, notamment les performances de MeteorJS.\\
			
			Si MeteorJS devait se montrer sous-performant, nous partirons sur du développement Android pur (Java).
			
		\subsection{Logo et nom}
		
		\subsection{Maquette}
		
		\subsection{Planification initiale}
			% faut-il la mettre ici ?
	
\end{document}
