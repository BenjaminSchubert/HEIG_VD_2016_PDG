\section{Presentation de l'Application}

Nous souhaitons faire une application mobile permettant à un groupe de se retrouver facilement dans un festival, une grande occasion réunissant beaucoup de personnes, etc.
 L'interface représenterait une carte ou une boussole, au choix, permettant de se rejoindre le plus efficacement possible.
 La confidentalité des données utilisateurs, notamment GPS seraient bien évidément au centre des préocupations et les utilisateurs auraient plein contrôle sur celle-ci.
 Il serait possible de faire des listes de groupes et définir des points de rencontre. 

\section{Caractéristiques de l'application}

\subsection{Gestion des contacts}

\item Il sera possible de créer des groupes.
\item ajouter des gens, les bloquer, reporter un compte douteux


\subsection{Mode de retrouvailles}

\item Il sera possible de trouver automatiquement le point de rencontre le plus prêt de tous les membres
\item Il sera possible de définir un point de rencontre, fixe ou sur une personne.
\item Il sera possible de traquer un membre du groupe


\subsection{Gestion des données personnelles}

\item un utilisateur peut être marqué comme visible par tous ses contacts pour un temps donné
\item un utilisateur peut donner une visibilité indéfinie pour certains membres
\item un utilisateur peut accepter au cas par cas les demandes de traquages



\section{Motivation}

Nous souhaitons apprendre de nouvelles technologies, notamment mobiles et mettre en oeuvre ce que nous voyons en cours concernant les interface hommes-machines, les systèmes mobiles, etc.

Nous souhaitons utiliser de nouvelles technologies que les mobiles récents permettent


\subsection{Données techniques}
\item nous souhaitons utiliser OpenStreetMap pour les données de géolocalisation
\item nous souhaitons utiliser MeteorJS pour le développement mobile
\item nous souhaitons utiliser typescript afin de découvrire un nouveau langage en vogue


